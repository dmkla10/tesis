\chapter{Revisión Bibliográfica}

\section {Inferencia Bayesiana}

\subsection{Definición}
Como mencionan Gelman en \cite{gelman2013bayesian}, la `inferencia Bayesiana es el proceso de ajustar un modelo de probabilidad a un conjunto de datos y resumir los resultados con una distribución de probabilidad sobre los parámetros del modelo y sobre cantidades no observadas como predicciones para nuevas observaciones'.\\

La primera pieza para hacer inferencia con este paradigma, es definir un modelo de probabilidad conjunta sobre los datos, y los parámetros que gobiernan la generación de los datos; esto es, definir:

\begin{center}
    $p(\theta, y) = p(\theta)p(y|\theta)$
\end{center}

Esto corresponde a definir un modelo de probabilidad generador de los datos, dados los valores de $\theta$, y una distribución \textit{a priori} de los parámetros de dicho modelo, la cuál luego será actualizada con la observación de los datos.\\

Para realizar la actualización de la distribución de $\theta$, utilizamos la \textit{regla de bayes}:

\begin{center}
    $p(\theta | y) = \dfrac{p(y|\theta)p(\theta)}{p(y)}$
\end{center}

Donde el factor $p(y)$ corresponde a la distribución marginal de los datos en nuestro modelo. Este valor en la práctica no se utiliza pues corresponde a una constante.\\

\subsection{Modelos Jerárquicos}

Como se menciona en el capítulo 5 de \cite{gelman2013bayesian}, existen muchos casos donde los parámetros de nuestros modelos están relacionados por la estructura del modelo, implicando que la distribución conjunta de dichos parámetros debería reflejar dicha conexión.\\

Por ejemplo, en el caso del crímen, diferentes sujetos tienen diferentes 

\section{Hipergrafos y Predicción de Hiperenlaces}

\subsection{Definición de hipergrafo}

Como mencionan Chen \& Liu \cite{Chen_2024}, los hipergrafos son una generalización de los grafos, donde los hiperenlaces, enlaces de los hipergrafos, pueden tener una cantidad arbitraria de nodos.
Un hipergrafo se define como \begin{math} H = \{V,E\}\end{math} donde $V=\{n_{1},n_{2},...,n_{n}\}$ es el conjunto de nodos y $E=\{e_{1},e_{2},\dots,e_{m}\}$ es el conjunto de hiperenlaces, donde cada hiperenlace es un conjunto de nodos, o sea, $e_{p} \subseteq E \; \; \forall \: p \in 1,\dots,m$.\\

Para representar matricialmente un hipergrafo, se utiliza una \textit{matríz de incidencia} $H \in \mathbb{R}^{n \times m}$, donde cada fila representa un vector, cada columna un hiperenlace. En esta notación, el valor de $H_{j,i}$ es 1 si el nodo $i$ pertenece al hipernlace $j$, y 0 si no.\\

Como ejemplo, a continuación mostramos una matriz de incidencia junto al hipergrafo que representa:\\

\begin{minipage}{.5\linewidth}
    \begin{center}
            \begin{blockarray}{ccccc}
            & $e_1$ & $e_2$ & $e_3$  \\
            \begin{block}{c(cccc)}
            $n_1$ & 1 & 0 & 0  \\
            $n_2$ & 0 & 1 & 1  \\
            $n_3$ & 1 & 0 & 0  \\
            $n_4$ & 1 & 1 & 0  \\
            $n_5$ & 0 & 1 & 1  \\
            $n_6$ & 0 & 1 & 0  \\
            \end{block}
            \end{blockarray}
                       
    \end{center}
    \end{minipage}%
    \begin{minipage}{.5\linewidth}
        \begin{center}
            Imagen
        \end{center}
\end{minipage}

Para nuestro caso, tenemos además un vector de atributos para cada nodo e hiperenlace, $w_i$ y $z_j$ respectivamente, con $i \in 1,\dots,n$ y $j \in 1,\dots,m$.

\subsection{Predicción de hiperenlaces}
En \cite{Chen_2024}, Chen \& Liu realizan un análisis sistemático de la literatura sobre predicción de hipergrafos hasta el año 2022. En este articulo, se plantea el problema de la predicción de hiperenlaces como el aprendizaje de la función $\Psi $ tal que para un hiperenlace potencial $e$ se logre:

\begin{center}
    \begin{math}
        \Psi (e) = 
        \begin{cases}
            \geq \epsilon &\text{si $e \in E$}\\
            < \epsilon &\text{si $e \notin E$}
        \end{cases}
    \end{math}
\end{center}

donde $\epsilon$ es un valor de corte para convertir un posible valor continuo de $\Psi$ en valor binario. Esto puede ser visto como la detección de hiperenlaces no observados, en el caso de hipergrafos estáticos, o como la predicción de hiperenlaces futuros, en el caso de hipergrafos dinámicos.\\

Notamos que este planteamiento formula el problema de predicción de hiperenlaces como un problema de clasificación, una extensión natural de la predicción de enlaces en redes. Sin embargo, este planteamiento sufre del problema de clases desbalanceadas de manera extrema, pues la cantidad de hiperenlaces posibles para un hipergrafo con n nodos es $2^n$ \cite{Prasanna_2020} \cite{Hwang_2022}. Para entender este problema, notemos que para un hipergrafo con 100 nodos, una cantidad bastante modesta, tenemos $1.27 \times 10^{30}$ posibles hiperenlaces.\\

Es por esto que para la predicción de hiperenlaces bajo esta formulación, se utiliza la técnica llamada \textit{negative sampling}, una técnica para... (leer el paper de negative sampling para comentar un poco)\\

Este enfoque, aunque ampliamente utilizado en la literatura sobre predicción de hiperenlaces, es criticada por Yu et al en \cite{Yu_2024}, mencionando que esta formulación asume que con un pequeño muestreo de los hiperenlaces negativos podemos generalizar la predicción sobre toda la población de hiperenlaces negativos.\\

Como alternativa, en \cite{Yu_2024} proponen un planteamiento donde: \textbf{dado} un hipergrafo $H = (V,E,X)$, con X la matriz de atributos de los nodos, un conjunto de nodos de consulta $Q \subset V$, un tamaño objetivo $s$, y una cantidad de soluciones objetivo $k$, \textbf{se busca obtener} hasta $k$ hiperenlaces que pertenezcan al conjunto de hiperenlaces verdaderos positivos para nuestro hipergrafo, \textbf{sujeto a} que el conjunto de nodos de consulta estén dentro de los hiperenlaces propuestos. (este parrafo está casi igual al del paper citado, está bien?)\\ 

Este planteamiento se acerca nuestro caso de estudio más que la formulación como clasificación.\\

Para nuestro caso, el planteamiento a resolver es: dado un hipergrafo $H=(V,E,W,Z)$, con $W$ la matriz de atributos de los nodos y $Z$ la matriz de atributos de los hiperenlaces, queremos obtener los nodos pertenecientes a un nuevo hiperenlace del cuál observamos sus atributos y, opcionalmente, un subconjunto de los nodos del hiperenlace nuevo. \\

En términos de nuestro caso de estudio particular, queremos, en base a la información histórica de asociaciones criminales, predecir para un nuevo crímen del cuál observamos los atributos y opcionalmente un subconjunto de los sujetos que participan en el, el conjunto de sujetos completo que cometió el crimen.