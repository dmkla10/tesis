% Template:     Tesis LaTeX
% Documento:    Archivo principal
% Versión:      3.4.2 (07/02/2025)
% Codificación: UTF-8
%
% Autor: Pablo Pizarro R.
%        pablo@ppizarror.com
%
% Manual template: [https://latex.ppizarror.com/tesis]
% Licencia MIT:    [https://opensource.org/licenses/MIT]

% CREACIÓN DEL DOCUMENTO
\documentclass[
	spanish, % Idioma: spanish, english, etc.	
	letterpaper, oneside
]{book}

% INFORMACIÓN DEL DOCUMENTO
\def\documenttitle {MODELO PROBABILÍSTICO BAYESIANO DE INFERENCIA DE HIPERENLACES EN REDES CRIMINALES}
\def\documentsubtitle {}
\def\degreetitle {
	TESIS PARA OPTAR AL GRADO DE MAGÍSTER EN GESTIÓN DE OPERACIONES
	\bigbreak\vspace{0.3cm}
	MEMORIA PARA OPTAR AL TÍTULO DE INGENIERO CIVIL INDUSTRIAL
}

\def\universityname {Universidad de Chile}
\def\universityfaculty {Facultad de Ciencias Físicas y Matemáticas}
\def\universitydepartment {Departamento de Ingeniería Industrial}
\def\universitydepartmentimage {departamentos/uchile2}
\def\universitydepartmentimagecfg {height=3cm}
\def\universitylocation {Santiago de Chile}

% INTEGRANTES, PROFESORES Y FECHAS
\def\documentauthor {DIMITRI LOAIZA ARAYA}
\def\documentdate {2025}

\def\portrait {
	\begin{center}
	\vspace{1.5cm} ~ \\
	\MakeUppercase{\textbf{\documenttitle}} ~ \\
	\vspace{1.5cm}
	\MakeUppercase{\degreetitle} ~ \\
	\vfill
	\begin{tabular}{c}
		\MakeUppercase{\textbf{\documentauthor}} \\ \\
		\vspace{1.0cm} \\
		PROFESOR GUÍA: \\
		RICHARD WEBER HAAS \\
		\vspace{0.5cm} \\
		MIEMBROS DE LA COMISIÓN: \\
		PROFESOR 2 \\
		PROFESOR 3 \\
		\vspace{0.5cm} \\
		Este trabajo ha sido parcialmente financiado por: \\
		NOMBRE INSTITUCIÓN \\
		\vspace{0.5cm} \\
		\MakeUppercase{\universitylocation} \\
		\MakeUppercase{\documentdate}
	\end{tabular}
	\end{center}
}
\def\abstracttable {
	\begin{tabular}{l}
		RESUMEN DE LA MEMORIA PARA OPTAR \\
		AL TÍTULO DE MAGÍSTER EN CIENCIAS \\
		DE LA INGENIERÍA \\
		POR: \MakeUppercase{\documentauthor} \\
		FECHA: \MakeUppercase{\documentdate} \\
		PROF. GUÍA: NOMBRE PROFESOR
	\end{tabular}
}

% IMPORTACIÓN DEL TEMPLATE
\input{template}

\usepackage{algorithm}
\usepackage{algpseudocode}
\usepackage{tikz}
\usetikzlibrary{bayesnet}

% INICIO DE LAS PÁGINAS
\begin{document}

% PORTADA
\templatePortrait

% CONFIGURACIÓN DE PÁGINA Y ENCABEZADOS
\templatePagecfg

% RESUMEN O ABSTRACT
\begin{abstractd}
	Hola buenas tardes a todos este es el abstract, espero que les guste.
\end{abstractd}

% DEDICATORIA
\begin{dedicatory}
	Una frase de dedicatoria, \\
	pueden ser dos líneas. \\
	~ \\
	\textbf{Saludos}
\end{dedicatory}

% AGRADECIMIENTOS
\begin{acknowledgments}
	Muchas gracias a todos.
\end{acknowledgments}

% TABLA DE CONTENIDOS - ÍNDICE
\templateIndex

% CONFIGURACIONES FINALES
\templateFinalcfg

% ======================= INICIO DEL DOCUMENTO =======================

\chapter{Introducción}

\section{Motivación}

El estudio de asociaciones o agrupaciones entre individuos es un tema de relevancia el análisis de redes sociales, con aplicaciones en diversas áreas, desde el marketing hasta la criminología. Cuando las asociaciones son de una cantidad variable de sujetos, es natural modelar el fenómeno como un hipergrafo: una generalización de los grafos donde los enlaces son de dos o más nodos.\\

En este contexto, un problema que ha ganado interés académico en los últimos años es la predicción de hiperenlaces; esto es, evaluar y proponer cuáles son los hiperenlaces más probables de de existir, ya sean futuros o faltantes, dados los patrones de asociación de los nodos en los hiperenlaces observados.\\

Nuestro interés por este problema nace desde un caso específico: la inferencia de participantes en un delito para el apoyo a la investigación penal en el contexto del proyecto Fiscal Heredia.\\

En la próxima sección daremos contexto sobre las aplicaciones de inteligencia artificial en la Fiscalía Nacional Chilena en el marco de Fiscal Heredia, para luego proponer un conjunto de características deseadas en un modelo de redes sociales para este contexto.

\section{Fiscal Heredia}

Explicar el origen y las diferentes aplicaciones.\\

Dado lo anterior, las características deseables del modelo planteado son:
\begin{enumerate}
\item Capacidad de aprender las preferencias de asociación de los nodos respecto a las características de la asociación (atributos de los hiperenlaces) como respecto a las características de los nodos con quienes se asocia.
\item Integración de un mecanismo de llegada de nuevos sujetos al mercado criminal a través de su participación en delitos con nodos ya pertenecientes a la red.
\item Capacidad de realizar inferencia sobre estos sujetos nuevos, los cuáles naturalmente tendrán poco historial para ajustar sus preferencias.
\item Capacidad de explicitar la incertidumbre sobre las inferencias realizadas, a manera de entregar información completa y transparente que apoye mejor el análisis criminal
\end{enumerate}


\section{Objetivos}

Los objetivos para este trabajo de tesis son:\\

Objetivo General:
\begin{itemize}
    \item Diseñar, desarrollar y evaluar un modelo bayesiano de inferencia de hiperenlaces que cumpla con las características deseables para su aplicación en el contexto de Fiscalía.\\
\end{itemize}


Objetivos Específicos:
\begin{itemize}
    \item Proponer un diseño para el modelo basado en la literatura y los requisitos de la aplicación.
    \item Desarrollar el modelo en software especializado de manera que sea computacionalmente factible.
    \item Evaluar el rendimiento del modelo con conjuntos de datos de crímen y de marketing para explorar sus posibles usos.
    \item Proponer una breve metodología para su uso.
\end{itemize}

\chapter{Revisión Bibliográfica}

\section {Inferencia Bayesiana}

\subsection{Definición}
Como mencionan Gelman en \cite{gelman2013bayesian}, la `inferencia Bayesiana es el proceso de ajustar un modelo de probabilidad a un conjunto de datos y resumir los resultados con una distribución de probabilidad sobre los parámetros del modelo y sobre cantidades no observadas como predicciones para nuevas observaciones'.\\

La primera pieza para hacer inferencia con este paradigma, es definir un modelo de probabilidad conjunta sobre los datos, y los parámetros que gobiernan la generación de los datos; esto es, definir:

\begin{center}
    $p(\theta, y) = p(\theta)p(y|\theta)$
\end{center}

Esto corresponde a definir un modelo de probabilidad generador de los datos, dados los valores de $\theta$, y una distribución \textit{a priori} de los parámetros de dicho modelo, la cuál luego será actualizada con la observación de los datos.\\

Para realizar la actualización de la distribución de $\theta$, utilizamos la \textit{regla de bayes}:

\begin{center}
    $p(\theta | y) = \dfrac{p(y|\theta)p(\theta)}{p(y)}$
\end{center}

Donde el factor $p(y)$ corresponde a la distribución marginal de los datos en nuestro modelo. Este valor en la práctica no se utiliza pues corresponde a una constante.\\

\subsection{Modelos Jerárquicos}

Como se menciona en el capítulo 5 de \cite{gelman2013bayesian}, existen muchos casos donde los parámetros de nuestros modelos están relacionados por la estructura del modelo, implicando que la distribución conjunta de dichos parámetros debería reflejar dicha conexión.\\

Por ejemplo, en el caso del crímen, diferentes sujetos tienen diferentes 

\section{Hipergrafos y Predicción de Hiperenlaces}

\subsection{Definición de hipergrafo}

Como mencionan Chen \& Liu \cite{Chen_2024}, los hipergrafo son una generalización de los grafos, donde los hiperenlaces pueden tener una cantidad arbitraria de nodos.
Matemáticamente, se define como \begin{math} H \in \{V,E\}\end{math} donde $V=\{n_{1},n_{2},...,n_{n}\}$ es el conjunto de nodos y $E=\{e_{1},e_{2},\dots,e_{m}\}$ es el conjunto de hiperenlaces, donde cada hiperenlace es un subconjunto de nodos, o sea, $e_{p} \subset E \; \; \forall \: p \in 1,\dots,m$.\\

Para representar matricialmente un hipergrafo, se utiliza una \textit{matríz de incidencia} $H \in \mathbb{R}^{n \times m}$, donde cada fila representa un vector, cada columna un hiperenlace. En esta notación, el valor de $H_{j,i}$ es 1 si el nodo $i$ pertenece al hipernlace $j$, y 0 si no.\\


\subsection{Definición de predicción de hiperenlaces}
En este mismo artículo, se plantea el problema de la predicción de hiperenlaces como el aprendizaje de la función $\Psi $ tal que para un hiperenlace potencial $e$ se logre:

\begin{center}
    \begin{math}
        \Psi (e) = 
        \begin{cases}
            \geq \epsilon &\text{si $e \in E$}\\
            < \epsilon &\text{si $e \notin E$}
        \end{cases}
    \end{math}
\end{center}

donde $\epsilon$ es un valor de corte para convertir un posible valor continuo de $\Psi$ en valor binario.\\

\subsection{Modelos de predicción de hiperenlaces}

El tema principal de la publicación es una revisión sistemática de la literatura sobre predicción de hiperenlaces; esta será nuestra fuente inicial para entender el estado del arte.\\

Se propone una taxonomía que clasifica los modelos en cuatro categorías: (1) métodos basados en similitud, (2) basados en probabilidad, (3) basados en optimización de matrices y (4) basados en aprendizaje profundo.\\

Dentro de cada categoría también se divide entre métodos directos e indirectos, siendo indirectos aquellos que son inicialmente propuestos como modelos de predicción de enlaces o clusterización, y que pueden ser modificados para hacer predicción de hiperenlaces; y directos aquellos que son desarrollandos específicamente para predicción de hiperenlaces.\\

Como mencionamos en la introducción, son de nuestro interés los modelos probabilísticos bayesianos; de este tipo de modelos, se menciona un modelo indirecto basado en \cite{NIPS2005_182e6c2d}.\\

\chapter{Modelo Propuesto}

\section{Modelo Generativo Subyacente}

Para plantear nuestro modelo, asumimos un modelo generativo subyacente, el cuál no nos interesa ajustar ni modelar en detalle, pero que es útil para el planteamiento del nuestro modelo de predicción de enlaces.\\

El algoritmo generativo de la hiperred se define como:

\begin{algorithm}
\caption{Hypergraph Generative Algorithm}\label{alg:cap}
\begin{algorithmic}
\Require $H_{0}$,  $p(e_{t+1}, a_{e_{t+1}}) = f(e_{t+1}, a_{e_{t+1}}|H_{t}, \Theta)$
\State $t \gets 0$
\While{$t \leq T$}

Sample the next hyperedge from $p(e_{t+1}, a_{e_{t+1}})$
\State $t \gets t+1$

\EndWhile
\end{algorithmic}
\end{algorithm}

Notación:\\

Los nodos de cada hiperenlace t se dividen entre nodos que ya eran parte de la red y nodos nuevos:\\

$e_t =  e_{nuevos_t} \bigcup e_{antiguos_t}$\\

Los nodos de cada hiperenlace nuevo se dividen entre nodos conocidos y nodos desconocidos:\\

$e_{t+1} = e_{conocidos_{t+1}} \bigcup e_{desconocidos_{t+1}}$\\

Uniendo ambas notaciones, tenemos que un hiperenlace nuevo se descompone en:\\

$e_{t+1} = e_{antiguos, desconocidos_{t+1}} \bigcup e_{antiguos, conocidos_{t+1}} \bigcup e_{nuevos, conocidos_{t+1}} \bigcup e_{nuevos, desconocidos_{t+1}}$\\

$X_{S}$: Matriz de atributos de los nodos del conjunto S.\\

$p(e_{t+1}, a_{e_{t+1}}, X_{e_{nuevos_{t+1}}}   ) = f(e_{t+1}, a_{e_{t+1}}|H_{t}, \Theta)$\\

(1) $p(|e_{t+1}| = k \;|\; H_{t}, a_{e_{t+1}}, e_{conocidos_{t+1}}, \gamma)$\\

(2) $p(|e_{nuevos_{t+1}}| = l \;|\; H_{t}, |e_{t+1}|, a_{e_{t+1}}, e_{conocidos_{t+1}}, \theta)$\\

(3) $p(n_{i} \in e_{antiguos,desconocidos_{t+1}} \;|\; H_{t}, |e_{t+1}|, a_{e_{t+1}}, e_{conocidos_{t+1}}, \beta)$\\

(4) $p(X_{e_{nuevos,desconocidos_{t+1}}} \;|\; H_{t}, |e_{t+1}|, a_{e_{t+1}}, \lambda)$\\


\bibliography{library}

\end{document}